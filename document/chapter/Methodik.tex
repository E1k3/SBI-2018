\chapter{Methodik}
\label{ch:methodik}

FMOE fällt in die dritte Kategorie und unterscheidet sich von anderen Verfahren derselben Kategorie hauptsächlich durch die Verwendung von FM-Index und damit verbundenen Optimierungen.

Im Folgenden werden daher zuerst die Datenstrukturen behandelt.

\section{Datenstrukturen}
\label{sec:datenstrukturen}

Der FM-index bietet Möglichkeiten zur schnellen Patternsuche bei sehr geringer Speicherkomplexität.
Die Funktionsweise und die Grunde für die Resourcensparsamkeit lassen sich am besten im Vergleich zum Suffix-Array erläutern.

\begin{tabular}{ | c | c | c | c | c | c | c | }
        \hline
        a & b & c & d & e & f & g \\
        \hline
        b & c & d & e & f & g &   \\
        \hline
        c & d & e & f & g &   &   \\
        \hline
        d & e & f & g &   &   &   \\
        \hline
        e & f & g &   &   &   &   \\
        \hline
        f & g &   &   &   &   &   \\
        \hline
        g &   &   &   &   &   &   \\
        \hline
\end{tabular}
