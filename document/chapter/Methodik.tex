\chapter{Methodik}
\label{ch:methodik}

Im Methodik-Kapitel werden die mathematischen Ausführungen des Verfahrens bzw. des Algorithmus vorgestellt, jedoch technische Details zur Umsetzung und Implementierung (falls nötig) auf ein darauffolgendes Kapitel verlagert.


\section{Mathematische Notation}
\label{s:notation}

Mathematische Formeln können mittels der \verb+\begin{align}...\end{align}+ Umgebung gesetzt werden:

\begin{align}
f(n) & =
	\begin{cases}
		n/2, & \text{wenn }n\text{ gerade,}\\
		3n+1, & \text{wenn }n\text{ ungerade.}
	\end{cases}
\label{eq:f} \\
%
g(n) & = \frac{n}{2} \label{eq:g}
\end{align}

\section{Algorithmus}
\label{s:algorithmus}

Eigene Algorithmen beschreibt man am Besten mit Hilfe von Pseudo-Code und dem Paket \verb+algorithm+.

\begin{algorithm}
\caption{Algorithmus}
\label{alg:alg}
\begin{algorithmic}
\algsetup{indent=2em}

\REQUIRE Argument $n\in\mathbb{N}$
\STATE $a = 0$
\FOR{ $i=0,\dots,n$}
	\STATE $a = a + 1$
\ENDFOR
\RETURN $a$
\end{algorithmic}
\end{algorithm}