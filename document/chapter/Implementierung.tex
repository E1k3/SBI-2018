\chapter{Implementierung}
\label{ch:implementierung}

In diesem Kapitel werden Technische Details, wie zum Beispiel die Verwendete Laufzeitumgebung, Laufzeitanalyse oder wichtige Implementierungsdetails behandelt.
Dabei sollte beachtet werden inwiefern diese Angaben für die Arbeit von Bedeutung sind. Die Namen einzelner Funktionsaufrufe oder Klassendiagramme sind zum Beispiel im Allgemeinen für den Leser uninteressant.

\section{Quelltext-Abschnitte}
Eigene Auszüge aus dem Quelltext bindet man am Einfachsten mit dem Befehl \verb+\lstinputlisting{}+ ein. Das Ergebnis ist in \ref{fig:code} zu sehen.
Optionen für die Sprache, Tab-Breite etc. von \verb+\lstinputlisting+ können auch am Anfang des Dokuments mit \verb+\lstset{}+ gesetzt werden.
Das Paket erlaubt mit \verb+firstline=...+ und \verb+lastline=...+ auch die Einbindung von einzelnen Zeilen einer Datei.

%Auch möglich: Kapselung in \begin{figure} ... \end{figure} für flexible Positionierung
\lstinputlisting[language=C,tabsize=2,frame=single,caption={Codebeispiel}]{src/myTest.m}
\label{fig:code}


