\chapter{Ergebnisse}
\label{ch:ergebnisse}

FMOE wurde in dem behandelten Paper mit sechs anderen Korrekturalgorithmen vergleichen (siehe Tabelle \ref{tbl:comparison}).
Alle zählten entweder zu den überlappungsbasierten oder k-mer Frequenzspektrum Verfahren.

\begin{figure}[h]
	\begin{center}
		\small
		\begin{tabular}{ c c c c c }
			Algorithmus & Erscheinungsjahr & Überlappungsbasiert & k-mer Spektrum & Read Qualität \\
			\hline
			FMOE & 2017 & x &   & Nein \\
			Coral & 2011 & x &   & Ja \\
			Karect & 2015 & x &   & Ja \\
			QuorUm & 2015 &   & x & Ja \\
			RACER & 2013 &   & x & unbkannt \\
			BLESS2 & 2016 &   & x & Ja \\
			SGA & unbekannt & x & x & unbekannt \\
			\hline
		\end{tabular}
		\caption{Übersicht der verglichenen Fehlerkorrekturalgorithmen. Quellen \cite{ComparisonThesis} \cite{Coral} \cite{Karect} \cite{QuorUm}}
		\label{tbl:comparison}
	\end{center}
\end{figure}

Als Testdaten wurden vier Datensätze aus der GAGE-b Datenbank verwendet.
Die GAGE-b Datenbank enthält Daten aus echten Sequenzierungen, durchgeführt mit den verbreitetsten Sequenzierverfahren.
Das Ziel der Datenbank ist es, Sequenzierungs-, Assembly- und Korrekturverfahren mit realistischen Anwendungsfällen vergleichbar zu machen.

Die Größe der Datensätze variert zwischen 4,6Mb und 100,2Mb mit Readlängen von 110bp bis 251bp.

Die Güte eines Verfahrens wird im Allgemeinen durch die Korrekturleistung (Menge korrigierter Basen) und die Genauigkeit (Anteil richtiger Korrekturen) bestimmt.

Zu erwarten war, dass die überlappungsbasierten Verfahren - und damit auch FMOE - bei den größeren Read Längen eine höhere Korrekturleistung aufweisen und grundsätzlich, aufgrund des Alignments, deutlich langsamer sind als k-mer Frequenzspektrum basierte Verfahren.

Die Ergebnisse scheinen die Annahme mindestens für FMOE zu bestätigen, denn oft haben Coral oder Karect und besonders häufig hat FMOE relativ hohe Korrekturleistungen bei den Datensätzen mit 251bp Readlänge.
Die Korrekturleistungen bei Datensätzen mit kurzer Readlänge sind kaum voneinander zu unterscheiden.

Bis auf einen Aussetzer von Coral bewegt sich die Identität der korrigierten Datensätze mit dem korrekten Genom in allen Fällen deutlich über den Rohdaten.

Die Laufzeit der k-mer basierten Methoden beträgt, wie erwartet nur einen Bruchteil der Laufzeit der überlappungsbasierten Verfahren.
Der Unterschied zwischen Coral auf der einen und Karect und FMOE auf der anderen Seite ist allerdings enorm.
Die längste Laufzeit von Coral beträgt 43 Stunden, während Karect und FMOE für denselben Datensatz ca. 2 Stunden benötigen.
