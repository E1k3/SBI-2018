\chapter{Diskussion und Fazit}
\label{ch:diskussion}

Das behandelte Paper stellt FMOE im Detail vor und erläutert die Funktionsweise an vielen Stellen sehr anschaulich.

An einigen Stellen fehlen allerdings Details:
\begin{description}
	\item[Qualität der Reads]\hfill\\
		Aktuelle Sequenzierungsverfahren stellen nicht nur die rohen Reads zur Verfügung, sondern auch weitere relevante Daten.
		Unter anderem ist im Normalfall ebenso eine Qualitätsmetrik für jeden einzelnen Read oder sogar jede Base enthalten.
		Diese ermöglicht es, die Fehlerwahrscheinlichkeit der Basen des Reads einzuschätzen und so Basen mit hohen Fehlerquoten bei der Korrektur oder Statistik schwächer zu gewichten.
		%cite blog bfc

		Das Paper gibt keinen Aufschluss darüber, ob FMOE diese Daten ignoriert, andere Verfahren nutzen Sie allerdings erfolgreich \cite{ReadQuality} \cite{BfcPaper} (siehe \ref{tbl:comparison}).

	\item[Stichprobemgröße]\hfill\\
		Die Größe der k-mer Stichprobe ist laut dem Paper konstant vorgegeben.
		Das kann bei großen Datensätzen für Probleme sorgen, da eine relativ kleine Stichprobe die Daten nicht mehr ausreichend repräsentieren kann.

		Ob FMOE eine Möglichkeit bietet, diese Größe anzupassen oder sie dynamisch angepasst wird, ist nicht durch das Paper ersichtlich.
	
	\item[Einstellungen der konkurrierenden Algorithmen]\hfill\\
		Das Paper enthält keine Informationen über die verwendeten Einstellungen bei den Vergleichsverfahren.

		Das SGA Paket bietet mehrere Fehlerkorrekturalgrithmen an.
		%TODO:cite comparison master thesis
		Hier ist nicht klar, welches dieser Verfahren angewendet wurde.

	\item[Darstellung der Ergebnisse]\hfill\\
		Die Ergebnisse des Vergleichs befinden sich in vielen Fällen in den letzten zwei Nachkommastellen und sind daher selbst bei so geringen Mengen schwer zu überblicken.

		An dieser Stelle währe eine andere Skalierung oder eine Visualisierung durch Graphen übersichtlicher gewesen.
\end{description}

Abgesehen davon macht das Paper aber einen guten Eindruck.

Besonders die Verwendung von GAGE-b Datensätzen für den Vergleich der Verfahren ist lobenswert, da hier auf synthetische Datensätze verzichtet wurde.
